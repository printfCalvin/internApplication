\documentclass[12pt]{article}
\usepackage{fullpage,url,amssymb,epsfig,color,xspace,tikz,amsmath, amsthm}

\begin{document}
    
\begin{center}
    \textbf{\huge Tutorial 6}
\end{center}
\section{Midterm}
\begin{itemize}
    \item Any questions about midterm?
\end{itemize}
\section{BST Deletion}
\begin{itemize}
    \item Lazy deletion. Add a boolean "zombie" flag to each node 
    Set to false initially on insertion, set to true on "deletion".
    \begin{itemize}
        \item This works OK in short term, or if only few deletes, but
        performance degrades if tree is full of zombies
    \end{itemize}
    \item A better way:
    \begin{enumerate}
        \item Target node has no children.
        \begin{itemize}
            \item Just delete the target node and send parent ptr to nullptr 
        \end{itemize}
        \item Target node has only one child.
        \begin{itemize}
            \item Set the target node's parent to point to the only child. And delete thr target node.
        \end{itemize}
        \item Target node has two children.
        \begin{itemize}
            \item Find the replacement nodes
            \begin{itemize}
                \item Biggest key down the left subtree(rightmost key in left subtree), or 
                \item Smallest key down the right subtree(leftmost key in right subtree)
            \end{itemize}
            \item Connect replacement parent's to its only child
            \item Replace target node with replacement node
            \begin{itemize}
                \item Set replacement node's left / right ptrs to values of target's left / right ptrs, respectively
                \item Set target's parent to point to replacement node
            \end{itemize}
            \item Delete target node
        \end{itemize}
    \end{enumerate}
\end{itemize}

\section{BST Traversal}
\begin{itemize}
    \item Three kinds of traversal:
    \begin{enumerate}
        \item Pre-Order
        \item In-Order
        \item Post-Order
    \end{enumerate}
    \item a trik to traversal a BST
\end{itemize}

\section{Exercises}
\begin{enumerate}
    \item Write a function that prints out the contents of a BST using Pre-Order Traversal. 
    For simplicity assume the keys are ints.
    \item Write a function that prints out the contents of a BST using Post-Order Traversal. 
    For simplicity assume the keys are ints.
    \item Write a function that determines whether a Binary Tree(assume duplicated values are allowed)
     is a BST or not, returning true if it is and false otherwise.
\end{enumerate}
\end{document}