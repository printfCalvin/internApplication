\documentclass[12pt]{article}
\usepackage{fullpage,url,amssymb,epsfig,color,xspace,tikz,amsmath, amsthm}
\usepackage{graphicx}
\begin{document}

\begin{center}
    \textbf{\huge Tutorial 2}
\end{center}

\section{Shell Command Review}
\begin{itemize}
    \item \textbf{echo} - display a line of text

    \item \textbf{cd} - change directory
    \begin{itemize}
        \item with no directory or \textbf{\~} returns to your home directory
        \item with \textbf{-} returns to your previous directory
    \end{itemize}

    \item \textbf{ls} - view files in the current/specified directory
    \begin{itemize}
        \item with \textbf{-l} returns a long form listing of the files
        \item with \textbf{-a} returns all files(including hidden files)
        \item can combine multiple options, e.g. \textbf{ls -al}
    \end{itemize}

    \item \textbf{pwd} - print the absolute path to the current directory
    
    \item \textbf{mkdir} - create a new directory
    
    \item \textbf{cp, mv, rm} - copy, move/rename, remove
    \begin{itemize}
        \item with \textbf{-i} asks if you are sure
        \item with \textbf{-fr} for \textbf{rm} forces the shell to remove a file/directory
    \end{itemize}
    
    \item \textbf{cat/less} - print file contents to the terminal
    
    \item \textbf{diff} - compare two files
    
    \item \textbf{find} - search for files and directories
    \begin{itemize}
        \item with \textbf{name pattern} restricts file names to globbing pattern
    \end{itemize}

    \item \textbf{chgrp} - change the froup ownership of a file or directory
    \begin{itemize}
        \item with \textbf{-R} recursively modifies the group of a directory
    \end{itemize}

    \item \textbf{chmod} - change permissions of a file
    \begin{itemize}
        \item with \textbf{-R} recursively modifies the group of a directory
        \item \textbf{u} - user
        \item \textbf{g} - group
        \item \textbf{o} - other
        \item \textbf{a} - all
        \item \textbf{w} - write
        \item \textbf{r} - read
        \item \textbf{x} - execute
        \item \textbf{+} - adds permissions
        \item \textbf{-} - removes permissions
    \end{itemize}

    \item \textbf{man} - prints information about commands
\end{itemize}

\section{Alias}
Aliases give convenient names to commands you use a lot.\\
\textbf{Example:} \begin{verbatim}
    alias mar138="marmoset_submit cs138"
\end{verbatim} 

\section{Globbing}
\begin{itemize}
    \item \textbf{*} - matches 0 or more characters
    \item \textbf{?} - matches 1 character
    \item \textbf{\{\dots\}} - matches any alternative in the set
    \item \textbf{[\dots]} - matches 1 character in the set
    \item \textbf{[!\dots]} - matches 1 character not in the set
\end{itemize}

\section{Input/Output redirection}
\begin{itemize}
    \item \textbf{$<$} - reads input form a file reather than keyboard
    \item \textbf{$>, 1>, 2>$} - (over)writes output/error to a file rather than screen
    \item \textbf{$>>, 1>>, 2>>$} - appends output/error to a file rather than screen
\end{itemize}

\section{Piplining}
Pipes allow us to combine commands.\\
\textbf{Example: }suppose we want to count how many loops in all .cc files in \textbf{\~/cs138/a1} directory.\\
commands that we will be using:
\begin{itemize}
    \item \begin{verbatim} cat ~/cs138/a1/*.cc \end{verbatim} cats all .cc files' contents
    \item \begin{verbatim} egrep 'for|while' ~/cs138/a1/*.cc \end{verbatim} finds all lines with for or while
    \item \begin{verbatim} wc -l \end{verbatim} prints the number of lines
\end{itemize}
Put them together using pipeline, \begin{verbatim} cat ~/cs138/a1/*.cc | egrep 'for|while' | wc -l \end{verbatim}.\\
Or simplier, \begin{verbatim} egrep 'for|while' ~/cs138/a1/*.cc | wc -l \end{verbatim}

\section{Types of Quotes}
\subsection{Double Quotes}
Protects everything except doublequote, backquote, and \$VARS(suppresses globbing)
\subsection{Single Quotes}
Protects everything

\section{Exercise}
\begin{enumerate}
    \item Sort all files(including hidden files) lexicographically
    \item \begin{verbatim}
        echo *
        echo '*'
        echo "*"
        echo '${HOME}'
        echo "${HOME}"
        echo '"${HOME}"'
        echo "'${HOME}'"
    \end{verbatim}
    \item Store all the names of .cc files to assignlist.txt
    \item Find all .cc and .cpp files in current directory
\end{enumerate}

\section{Interesting commands}
\begin{verbatim}
    cowsay
    cmatrix 
\end{verbatim}
\end{document}